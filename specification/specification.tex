\documentclass[a4paper]{article}
\usepackage[margin=35mm]{geometry}
\usepackage[T1]{fontenc}
\usepackage{xcolor}
\usepackage{textcomp}
\usepackage{listings}
\usepackage[utf8]{inputenc}
\usepackage{lmodern}
\usepackage[english]{babel}
\usepackage{amsmath}
\usepackage{graphicx}
\usepackage{caption}
\usepackage{subcaption}
\usepackage{amsfonts}
\usepackage{enumitem}

%\usepackage{svg}
\usepackage{tikz}
\usetikzlibrary{automata,positioning}

\renewenvironment{abstract}
 {\small
  \begin{center}
  \bfseries \abstractname\vspace{-.5em}\vspace{0pt}
  \end{center}
  \list{}{
    \setlength{\leftmargin}{.5cm}%
    \setlength{\rightmargin}{\leftmargin}%
  }%
  \item\relax}
 {\endlist}

\begin{document}

{
\part*{\center Lightweight Crypto in Reverse}
}

\begin{abstract}
The aim of this thesis is to look at cryptography in a 
reversible computing environment. Reversibility works on the principle 
that no information can be destroyed during the computation, so any 
computation can be traced to the beginning. This feature has the 
implication that any residual information is either known (e.g. 
null-initialized memory stays zero at the end of the computation), or 
the encryption implementation can be potentially attacked by using the 
residual information to lower the searched entropy for brute-force 
decryption.

In an emerging classification of encryption algorithms, we 
consider \textit{lightweight encryption} algorithms for several 
reasons: reversible computation has a direct relation to low-power 
applications, and also the current breed of programming languages is 
still rather experimental; standard practices have not evolved yet, so 
preferably simpler algorithms shall be considered first.
\end{abstract}


\section*{Project Description}

With the ongoing boom of small devices, often related to terms like 
"Internet of Things" and "smart home", we see a new requirement in the 
area of security, particularly cryptography. These devices have too 
little processing power (e.g. due to small physical sizes or 
limitations in terms of electrical power consumption) that they cannot 
run a complex encryption scheme that is normally required by today's 
standards when, for example, browsing the web. These require what is 
now known as \textsc{lightweight cryptography} - a set of hashing 
algorithms, ciphers and other that are optimized for as little CPU 
processing as possible.

An emerging area in computer science is the concept of 
\textsc{reversible computing}, where programs can be executed in both 
directions of the program flow, forwards and backwards. Two languages 
that have existing interpreters are Janus (imperative, C-like) and RFun 
(functional, Haskell-like). Both languages are still experimental and 
have not yet established a programming culture or community and their 
features are still being extended. These facts may become an obstacle, 
and so this alone is interesting topic to explore.

However, the combination of cryptography and reversibility gives us the 
opportunity to look for weaknesses in the implementation of 
cryptographic functions. Reversibility implies that no piece of 
information is destroyed. Since reversible programs can be executed in 
both directions, this also means that no information can be created 
(otherwise it would be destroyed when run in the other direction). So, 
for example, when given a string and a key, encryption outputs an 
encrypted string and a key, but nothing else. This lack of additional 
information means there is no secondary forgotten output that could be 
used to weaken the encryption. Furthermore, as a nice bonus we get a 
decryption function just by reversing the program flow on the 
encryption.

\newpage

\section*{Learning Objectives}

At the end of the project I shall be able to:
\begin{enumerate}
\item write functionally correct Janus / RFun code
\item analyse difficulties in writing code in reversible languages from programmer's perspective
\item analyse potential issues when writing security-sensitive code in reversible languages
\item implement several cryptography-related functions; block ciphers and hash functions
\item assess the implementation from security perspective
\item suggest improvements in development of the reversible languages
\end{enumerate}

\nocite{*} % Print all references regardless of whether they were cited in the poster or not
\bibliographystyle{unsrt}
\bibliography{bibliography}

\end{document}
