\documentclass[a4paper]{article}
\usepackage[margin=33mm]{geometry}
\usepackage[T1]{fontenc}
\usepackage{xcolor}
\usepackage{textcomp}
\usepackage{listings}
\usepackage[utf8]{inputenc}
\usepackage{lmodern}
\usepackage[english]{babel}
\usepackage{amsmath}
\usepackage{graphicx}
\usepackage{caption}
\usepackage{subcaption}
\usepackage{amsfonts}
\usepackage{enumitem}

%\usepackage{svg}
\usepackage{tikz}
\usetikzlibrary{automata,positioning}

\renewenvironment{abstract}
 {\small
  \begin{center}
  \bfseries \abstractname\vspace{-.5em}\vspace{0pt}
  \end{center}
  \list{}{
    \setlength{\leftmargin}{.5cm}%
    \setlength{\rightmargin}{\leftmargin}%
  }%
  \item\relax}
 {\endlist}

\begin{document}

{
\part*{\center Lightweight Crypto in Reverse}
}

\begin{abstract} The aim of this thesis is to look at cryptography in a 
reversible computing environment. Reversibility works on the principle 
that no information can be destroyed during the computation, so any 
computation can be traced to the beginning. Also, in a clean setting no 
information is unintentionally copied. These features have the 
implication that any residual information is either known (e.g. 
null-initialized memory stays zero at the end of the computation), or 
the encryption implementation can be potentially attacked by using the 
residual information to lower the searched entropy for brute-force 
decryption.

In an emerging classification of encryption algorithms, we 
consider \textit{lightweight encryption} algorithms for several 
reasons: reversible computation has a direct relation to low-power 
applications, and also the current breed of programming languages is 
still rather experimental; standard practices have not evolved yet, so 
preferably simpler algorithms shall be considered first.
\end{abstract}


\section*{Project Description}

With the ongoing boom of small devices, often related to terms like 
"Internet of Things" and "smart home", we see a new requirement in the 
area of security, particularly cryptography. These devices have too 
little processing power (e.g. due to small physical sizes or 
limitations in terms of electrical power consumption) that they cannot 
run a complex encryption scheme that is normally required by today's 
standards when, for example, browsing the web; yet, they require at 
least some level of encryption due to the fact that a large portion of 
them will be transmitting data wirelessly. These require what is now 
known as \textsc{lightweight cryptography} - a set of hashing 
algorithms, ciphers and other that are optimized for as little CPU 
processing as possible.

An emerging area in computer science is the concept of 
\textsc{reversible computing}, where programs can be executed in both 
directions of the program flow, forwards and backwards. Two languages 
that have existing interpreters are Janus (imperative, C-like) and RFun 
(functional, Haskell-like). Both languages are still experimental and 
have not yet established a programming culture or community and their 
features are still being extended. To give an insight, the most complex 
algorithms implemented in Janus are either sorting or simple 
compression algorithms; for RFun, it is likely its self-interpreter. 
These facts alone make this topic interesting to explore.

However, the combination of cryptography and reversibility gives us the 
opportunity to look for some side-channel vulnerabilities. There are 
several of these, which in theory could be prevented in a reversible 
setting. Reversibility implies that no piece of information is 
destroyed. Since reversible programs can be executed in both 
directions, this also means that no information can be created, or 
added; only data transformation from input to output occurs. This shows 
a lack of direct information leakage - a one kind of side-channel 
vulnerability. Other theoretically possible channels are power 
consumption (requiring the actual reversible hardware), execution time 
(due to the "data-transformation" style of the implementation, the 
execution should be more-or-less free of branching) and possibly even 
electromagnetic radiation patterns (heat and radio) could be more 
predictable for any input. However, due to the lack of reversible 
hardware, we can focus only on those side-channels, that are at least 
somewhat independent of it. Therefore, in this thesis we focus only on 
the direct information leakage from correct algorithm implementation 
and execution timing\footnote{Note that even the same code can take 
different amounts of time to execute based on the input; such is the 
experience from complex processors like the x86 architecture. Thus, in 
this sense we can only analyse the actual code from the perspective of 
branching, when, e.g., dealing with special cases.}. Work in this area 
can also help in the future, when reversible hardware is accessible and 
the hardware-dependent side channels mentioned above can be tested. 
Furthermore, as a nice bonus of the reversible setting, we get a 
decryption function just by reversing the program flow on the 
encryption.

\newpage

\section*{Learning Objectives}

At the end of the project I shall be able to:
\begin{enumerate}
\item write functionally correct reversible code,
\item analyse difficulties in writing code in reversible languages from programmer's perspective,
\item analyse potential issues when writing security-sensitive code in reversible languages,
\item implement several cryptography-related functions; block ciphers and hash functions,
\item assess the implementation from security perspective, including an analysis of side-channels,
\item and suggest improvements in development of the reversible languages.
\end{enumerate}

\nocite{*} % Print all references regardless of whether they were cited in the poster or not
\bibliographystyle{unsrt}
\bibliography{bibliography}

\end{document}
