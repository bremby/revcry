
\documentclass[a4paper,10pt,openright]{memoir}

% Style of front page, title page, and stuff page
\usepackage{dikuReport}


% Packages
\usepackage[utf8]{inputenc}
\usepackage{latexsym}
\usepackage[T1]{fontenc}
\usepackage{cite}
\usepackage{pdfpages}
\usepackage{tikz}
\usepackage{fix-cm}
\usepackage{xcolor,calc}
\usepackage{graphicx}
\usepackage{amsbsy}
\usepackage{amssymb}
\usepackage{amsmath}
\usepackage{alltt}
\usepackage{float}
\usepackage{color}
\usepackage{relsize}
\usepackage{calc}
\usepackage{ifthen}
\usepackage{xspace}
\usepackage{todonotes}

%%%%% Make abbreviations emphasized.
\newcommand{\ie}{\emph{i.e.}\xspace}
\newcommand{\eg}{\emph{e.g.}\xspace}
\newcommand{\etc}{\emph{etc.}\xspace}
\newcommand{\vs}{\emph{vs.}\xspace}
\newcommand{\cf}{\emph{cf.}\xspace}
\newcommand{\viz}{\emph{viz.}\xspace}
\newcommand{\etal}{\emph{et~al.}\xspace}

\def\enc{\ensuremath{\mathit{enc}}}
\def\dec{\ensuremath{\mathit{dec}}}
\def\inv{\ensuremath{\mathcal{I}}}
\newcommand{\exe}[1]{\ensuremath{[\![#1]\!]}}

\newcommand{\name}[1]{\textsc{#1}}
\newcommand{\term}[1]{\textit{#1}}

% Include some layout setup.
%!TEX root = main.tex

% NO RESETTING OF FOOTNOTES AT CHAPER CHANGE
\usepackage{remreset}
\makeatletter\@removefromreset{footnote}{chapter}\makeatother

% Set the style of chapter heads

%\definecolor{TemplateColor}{rgb}{.75,.116,.60}  % KU Red
\definecolor{TemplateColor}{rgb}{.2941,.4549,.2359} % KU green

\newcommand{\pnumb}[1]{
    \begin{tikzpicture}
      \draw[fill,color=gray] (0,0) rectangle (5mm,5mm);
      \draw[color=white] (2.5mm,2.5mm) node { #1};
    \end{tikzpicture}
}    
\newcommand{\printchapternumm}{}
\makechapterstyle{combined}{
  \setlength{\midchapskip}{-60pt}
  \setlength{\afterchapskip}{2.5cm}
  \renewcommand*{\printchaptername}{}
  \renewcommand*{\chapnumfont}{\normalfont\bfseries\fontsize{45}{0}\selectfont}
  \renewcommand*{\printchapternum}{\flushright\chapnumfont\textcolor{TemplateColor}{\thechapter}}
  \renewcommand*{\chaptitlefont}{\normalfont\Huge\bfseries}
  \renewcommand*{\printchaptertitle}[1]{%
    \raggedright\chaptitlefont\parbox[t]{\textwidth-2cm}{\raggedright##1}}
}

% Normal page numbering layout. How is always looks
\newcommand{\normalPN}{\makeevenfoot{plain}{}{\hspace{2mm}\thepage}{}\makeoddfoot{plain}{}{\thepage}{}}

% Special page numbering layout that is used for papers. Numbers are written in a gray box.
\newcommand{\specialPN}{\makeevenfoot{plain}{}{\\~\\~\\\hspace{2mm}\pnumb{\thepage}}{}\makeoddfoot{plain}{}{\\~\\~\\\pnumb{\thepage}}{}}

% Only clearpage (not cleardoublepage) at new chapter.
\renewcommand{\clearforchapter}{\clearpage}


% Sets page numbering layout to normal
\normalPN

% Where graphic files are located
\graphicspath{{figures/}}

% Alters the margins of the pages such that text are _not_ centered. This makes better room for the glue bindings.
\addtolength{\foremargin}{-45pt}
\addtolength{\spinemargin}{45pt}
\checkandfixthelayout


\begin{document}

% Basic information
\thesistype{MSc thesis}
\thesiscomment{} % You can leave this blank
\title{Lightweight Crypto in Reverse}
\subtitle{Exploring lightweight cryptography in the context of reversible computing}
\author{Dominik Táborský}
\supervisor{Michael Kirkedal Thomsen, Ken Friis Larsen}
%\date{July 7, 2018} % Hand-in date <<<------------------------<<<-----------------<<<-----------------!!!!!! FILL IN !!!!<<<-------------
%\subject{The short description that is suitable for a database.} % This is not needed.

% Make the front page, title page, and other required information.
\pagestyle{plain}
\maketitle

% Start at page 3. I do not count the front page in the numbering.
\cleardoublepage
\pagenumbering{roman}
\setcounter{page}{3}

% English Abstract
\cleardoublepage
\pagestyle{plain}
\begin{abstract}
Abstract
\end{abstract}

% Danish abstract
% \clearpage
% \begin{resume}
% Dansk resum\'e
% \end{resume}

% Table of contents
\cleardoublepage
\chapterstyle{combined}
\tableofcontents*


% Starting the real text.
\cleardoublepage
\pagenumbering{arabic}
\setcounter{page}{1}


%\cftaddnumtitleline{toc}{chapter}{Preface}{}{}%
\cftaddnumtitleline{toc}{chapter}{Preface}{}{\arabic{page}}%
\chapter*{Preface}

Preface and acknowledgements if you like to have this.



\cleardoublepage
%\cftaddnumtitleline{toc}{chapter}{Preface}{}{}%
\cftaddnumtitleline{toc}{chapter}{List of Abbreviations}{}{\arabic{page}}%
\chapter*{List of Abbreviations}

....


%%%%%%%%%%%%%%%%%%%%%%%%%%%%%%%%%%%%%%%%%%%%%%%%%%%%%%%%%%%%%%%%%%%%%%%%%%%%%%%%
%%% INTRODUCTION
%%%%%%%%%%%%%%%%%%%%%%%%%%%%%%%%%%%%%%%%%%%%%%%%%%%%%%%%%%%%%%%%%%%%%%%%%%%%%%%%
\cleardoublepage

\chapter{Introduction}

%Overview of the context, current problems and difficulties, discussion 
%of current solutions, our solution and its short evaluation.

Cryptography in general has been with us for decades now, yet only 
recently we finally figured out how to reliably code and verify an 
actual implementation with the F*/HACL work \cite{HACL}. A similar 
approach takes the Vale project \cite{Vale}, designing a new language 
that utilizes either Dafny or F* for verification. These solutions 
verify not only the correctness of the code, but also prove the lack of 
state-based information leakage, and in the case of Vale, even 
timing-based leakage. These two are possibly the easiest side-channel 
vulnerabilities to exploit, especially given the fact that physical 
access to the computer is not necessary. Both of them can get difficult 
to control for since the compiler often provides no guarantees on the 
generated machine code. Similarly, programming languages do not provide 
semantics on low enough level to account for registers, caches and 
execution timing. Vale provides for these, which allows it to formally 
verify the cryptographic properties; on the other hand, it also makes 
the language difficult to learn and use due to the sheer breadth of 
details the programmer now has to consider, instead of leaving it to 
the compiler.

Our work aims for exploiting reversible computing to reason about 
state-based leakage. Informally, a reversible program has to follow a 
clearly defined path that transforms the input data to output data 
without losing information. Because of that lack of information loss, 
it can be described as an injective function. Encryption algorithms are 
bijective functions, implying they can be directly encoded in a 
reversible setting. In a formal description, we have

$$
\exe{\enc}(k,p) = (k,c) \,.
$$

The plaintext $p$ is directly transformed to a ciphertext $c$, while 
key $k$ remains the same. Note that the key cannot be omitted: either 
such the function would not be injective, thus irreversible, or the 
ciphertext would somehow contain the key, which would make the 
encryption useless by making the key public. 

In Janus we enforce this scheme by offloading any other data into 
temporary local variables. These variables are declared and defined at 
the top of a function and undeclared at the bottom, with their final 
value specified. This final value is necessary for reversibility: 
without it, such reversed function would have undefined initial values. 
Secondly, because we know the final value of each variable, that value 
can be subtracted from the variable, thus clearing it. This implies 
that local data, not part of the input and output, will be cleared and 
thus no state-based leakage is possible. We demonstrate this by 
translating a Janus program to C, annotating it using special keywords 
for a LLVM plugin and compiling into machine code that can be manually 
verified to clean up used memory.

Furthermore, thanks to reversibility we implement only the encryption 
functions. The decryption is then obtained by reversing the encryption. 
This decreases time spent on development, debugging, testing and 
showing the duality of encryption and decryption functions.

Our conclusion is that reversibility is a handy tool for developing not 
only cryptographic code, but also other injective functions. It gives 
no guarantees on avoiding timing vulnerabilities like Vale does, 
however it comes at a much lower cost of development.



The reason for 
vulnerabilities based on the implementation and not the mathematical 
basis of the algorithm is usually performance, simplicity, inexperience 
and even language and/or compiler support for low-level semantics.






\section{Overview of the chapters}


\chapter{Analysis}

%In this chapter we look at lightweightedness in cryptography, consider 
%fundamental properties of reversibility, analyze properties of our 
%solution{ and define terminology}.

\section{Symmetric and asymmetric cryptography}

Cryptographic algorithms can be defined as functions that take two 
parameters, a key and some data, perform either encryption or 
decryption and return the key and the modified data. Normally the key 
in the result would be omitted, however it is useful to recognize the 
fact that the key is not altered in any way. This will become useful 
later on.

Symmetric cryptography differs from the asymmetric kind by its use of 
only a single key for both encryption and decryption. These are the 
classic algorithms like AES, RC5 and Chacha20. They also tend to be 
much more efficient performance-wise than their asymmetric siblings. 
This efficiency also lead to establishing the subarea of 
\term{lightweight cryptography}. Algorithms classified as 
\term{lightweight} are generally good candidates for use in 
\term{Internet of Things (IoT)} devices, which often have low 
computational performance, need to limit power usage, have strict 
memory requirements, or even may be required to react quickly. A 
special kind of symmetric cryptographic algorithms are stream ciphers, 
which only generate seemingly random output which is then combined with 
the input data (plaintext or ciphertext) using some self-inverting 
operation, typically XOR. Therefore stream ciphers only implement this 
"random" output and not individual encryption and decryption functions, 
since they are trivial and equivalent.

Asymmetric cryptography, also known as \term{Public Key Cryptography 
(PKI)}, uses one key for encryption (a \term{public key}) and another 
for decryption (a \term{private key}). Its security relies on some hard 
mathematical problems like integer factorization in the case of RSA. If 
that could somehow be performed as fast as the factor multiplication, 
it would render the whole algorithm broken. It does not achieve the 
same levels of performance \todo{citation needed}. For that reason it 
is not considered lightweight. It can, however, be used for so-called 
hybrid encryption schemes, where PKI is used to exchange the secret key 
for some symmetric algorithm.

\subsection{Definition}

Formally we use the following notation to describe a function 
\enc that fits the description above:

\begin{align*}
\enc_S(k,p) = (k,c) \\
\dec_S(k,c) = (k,p) 
\end{align*}

This suffices to describe symmetric cryptographic algorithms including 
stream ciphers where $\enc = \dec$.

For asymmetric algorithms we have to differentiate between public and 
private keys:

\begin{align*}
\enc_A(k_\text{public},p) = (k_\text{public},c) \\
\dec_A(k_\text{private},c) = (k_\text{private},p) 
\end{align*}

However, notice one major difference between the encryption functions 
$\enc_S$ and $\enc_A$: the resulting pair $(k, c)$ of $\enc_S$ contains 
all necessary information to decrypt the ciphertext back using 
$\dec_S$. That is not the case with $\enc_A$: having access to 
$(k_\text{public},c)$ does not suffice for decryption, since we are 
missing the private key $k_\text{private}$. This is an important 
distinction that we will explore later in \ref{asym_rev}. In short, the 
implication is that the definition of $\enc_A$ is not as complete as 
$\enc_S$.

\section{Side-channel vulnerabilities in cryptographic code}

The mathematical basis upon which cryptographic algorithms are built is 
only one half of the story when it comes to evaluating data security. 
The other half are their implementations, which have a long history of 
hidden vulnerabilities in various forms. These are called 
\term{side-channel vulnerabilities}, \term{side-channel information 
leakages} or just \term{side-channels} for short. These are classified 
into several groups depending on the type of the attack. The easiest 
(in terms of required prerequisites) are state-based and timing-based 
leakages, since they can be performed remotely and do not require any 
specialized hardware. Some other side-channels are power consumption 
and electromagnetic radiation analyses.

\todo{examples?}
\todo{how to defend against them?}

\section{Reversibility and its implications}

\todo{intro}

\subsection{Definition}

\subsection{Expressive power}

\subsection{Reversing asymmetric cryptography}
\label{asym_rev}

When we formally defined functions $\enc_S$ and $\enc_A$, we noticed 
the distinction in what information they give as a result. After 
looking at reversibility and seeing what it provides for symmetric 
cryptography, naturally it may seem like reversibility breaks 
asymmetric cryptography since an attacker knows both values, the public 
key and the ciphertext. We also know that reversibility does not 
increase expressive power of a language, it is just as powerful as a 
general Turing machine. So where is the problem?

The problem is that our definition was not complete. If the pair of the 
public key and the ciphertext contained all the information, then the 
algorithm would not provide any security. The security is exactly in 
the fact that the attacker has to search for some hidden information 
that is not available to them. One way of looking at this is simply 
considering multiplication: multiplying 3 and 4 has only one result, 
but if we know the result is 12, we do not know what the factors were. 
If we knew one of them was 3, then we would have all the information 
and could just calculate the other one. With irreversible setting this 
is not a problem, we delete information all the time, but that is not 
an option with reversible languages.

The correct and complete definition of $\enc_A$ would then be:

\begin{align*}
\enc_A(k_\text{public},p) = (k_\text{public}, p, c) \\
\end{align*}

Of course the plaintext is not transmitted or shared, only the 
ciphertext is. This definition avoids information loss and can be 
directly reversed by simply uncalculating \todo{define "uncalculating"} 
the ciphertext. In the case of the RSA algorithm, the public key has 
two components which are combined with the plaintext input in a 
loss-ful way. How this operation works is well-understood, out-of-scope 
of this thesis and the details are not important for us here.

\subsection{Data transformation}

\subsection{Bennett's tricks}

\subsection{Implementations}

\subsubsection{Reversible languages}

\subsubsection{Reversible hardware}

\section{Goals}


%\section{Terminology}

%directly reversible = an injective function -- wtf??? so how's Bennett's trick related to that?


\chapter{Design}

%This chapter defines the inputs and outputs for crypto algorithms, 
%expected behaviour and goals and finally methods of evaluating those 
%properties. 

+ use translation to other language for verification (e.g. Vale)

\chapter{Implementation}

%Here we show our implementations, prove their correctness and discuss them.

\chapter{Evaluation}

\section{design}
\section{implementations}
\section{experiments/testing}

\chapter{Reflections}

Evaluation of our solution in a broader context, usability in practice 
and viability of real-world implementation

\chapter{Related Work}

obvious

\chapter{Conclusion}

Discussion of results.


%%%%%%%%%%%%%%%%%%%%%%%%%%%%%%%%%%%%%%%%%%%%%%%%%%%%%%%%%%%%%%%%%%%%%%%%%%%%%%%%
%%% BIBLIOGRAPHY
%%%%%%%%%%%%%%%%%%%%%%%%%%%%%%%%%%%%%%%%%%%%%%%%%%%%%%%%%%%%%%%%%%%%%%%%%%%%%%%%

\cleardoublepage
\renewcommand{\sc}[1]{\textsc{#1}}
\nocite{*}
\bibliographystyle{acm}
\bibliography{bibliography}


%%%%%%%%%%%%%%%%%%%%%%%%%%%%%%%%%%%%%%%%%%%%%%%%%%%%%%%%%%%%%%%%%%%%%%%%%%%%%%%%
%%% Appendix
%%%%%%%%%%%%%%%%%%%%%%%%%%%%%%%%%%%%%%%%%%%%%%%%%%%%%%%%%%%%%%%%%%%%%%%%%%%%%%%%
\cleardoublepage
\appendix

% Updates section names to remove . (e.g. a section is named A1 instead of A.1)
\renewcommand{\thesection}{\thechapter\arabic{section}}

\chapter{The extra}


% To make sure that the backside print of the dissertation is a white page
\newpage ~
\thispagestyle{empty}



\end{document}
