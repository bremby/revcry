
\documentclass[a4paper,10pt,openright]{memoir}

% Style of front page, title page, and stuff page
\usepackage{dikuReport}


% Packages
\usepackage[utf8]{inputenc}
\usepackage{latexsym}
\usepackage[T1]{fontenc}
\usepackage{cite}
\usepackage{pdfpages}
\usepackage{tikz}
\usepackage{fix-cm}
\usepackage{xcolor,calc}
\usepackage{graphicx}
\usepackage{amsbsy}
\usepackage{amssymb}
\usepackage{amsmath}
\usepackage{alltt}
\usepackage{float}
\usepackage{color}
\usepackage{relsize}
\usepackage{calc}
\usepackage{ifthen}
\usepackage{xspace}
\usepackage{todonotes}

%%%%% Make abbreviations emphasized.
\newcommand{\ie}{\emph{i.e.}\xspace}
\newcommand{\eg}{\emph{e.g.}\xspace}
\newcommand{\etc}{\emph{etc.}\xspace}
\newcommand{\vs}{\emph{vs.}\xspace}
\newcommand{\cf}{\emph{cf.}\xspace}
\newcommand{\viz}{\emph{viz.}\xspace}
\newcommand{\etal}{\emph{et~al.}\xspace}

\newcommand{\name}[1]{\textsc{#1}}
\newcommand{\term}[1]{\textit{#1}}

% Include some layout setup.
%!TEX root = main.tex

% NO RESETTING OF FOOTNOTES AT CHAPER CHANGE
\usepackage{remreset}
\makeatletter\@removefromreset{footnote}{chapter}\makeatother

% Set the style of chapter heads

%\definecolor{TemplateColor}{rgb}{.75,.116,.60}  % KU Red
\definecolor{TemplateColor}{rgb}{.2941,.4549,.2359} % KU green

\newcommand{\pnumb}[1]{
    \begin{tikzpicture}
      \draw[fill,color=gray] (0,0) rectangle (5mm,5mm);
      \draw[color=white] (2.5mm,2.5mm) node { #1};
    \end{tikzpicture}
}    
\newcommand{\printchapternumm}{}
\makechapterstyle{combined}{
  \setlength{\midchapskip}{-60pt}
  \setlength{\afterchapskip}{2.5cm}
  \renewcommand*{\printchaptername}{}
  \renewcommand*{\chapnumfont}{\normalfont\bfseries\fontsize{45}{0}\selectfont}
  \renewcommand*{\printchapternum}{\flushright\chapnumfont\textcolor{TemplateColor}{\thechapter}}
  \renewcommand*{\chaptitlefont}{\normalfont\Huge\bfseries}
  \renewcommand*{\printchaptertitle}[1]{%
    \raggedright\chaptitlefont\parbox[t]{\textwidth-2cm}{\raggedright##1}}
}

% Normal page numbering layout. How is always looks
\newcommand{\normalPN}{\makeevenfoot{plain}{}{\hspace{2mm}\thepage}{}\makeoddfoot{plain}{}{\thepage}{}}

% Special page numbering layout that is used for papers. Numbers are written in a gray box.
\newcommand{\specialPN}{\makeevenfoot{plain}{}{\\~\\~\\\hspace{2mm}\pnumb{\thepage}}{}\makeoddfoot{plain}{}{\\~\\~\\\pnumb{\thepage}}{}}

% Only clearpage (not cleardoublepage) at new chapter.
\renewcommand{\clearforchapter}{\clearpage}


% Sets page numbering layout to normal
\normalPN

% Where graphic files are located
\graphicspath{{figures/}}

% Alters the margins of the pages such that text are _not_ centered. This makes better room for the glue bindings.
\addtolength{\foremargin}{-45pt}
\addtolength{\spinemargin}{45pt}
\checkandfixthelayout


\begin{document}

% Basic information
\thesistype{MSc thesis}
\thesiscomment{} % You can leave this blank
\title{Lightweight Crypto in Reverse}
\subtitle{Exploring lightweight cryptography in the context of reversible computing}
\author{Dominik Táborský}
\supervisor{Michael Kirkedal Thomsen, Ken Friis Larsen}
%\date{July 7, 2018} % Hand-in date <<<------------------------<<<-----------------<<<-----------------!!!!!! FILL IN !!!!<<<-------------
%\subject{The short description that is suitable for a database.} % This is not needed.

% Make the front page, title page, and other required information.
\pagestyle{plain}
\maketitle

% Start at page 3. I do not count the front page in the numbering.
\cleardoublepage
\pagenumbering{roman}
\setcounter{page}{3}

% English Abstract
\cleardoublepage
\pagestyle{plain}
\begin{abstract}
Abstract
\end{abstract}

% Danish abstract
% \clearpage
% \begin{resume}
% Dansk resum\'e
% \end{resume}

% Table of contents
\cleardoublepage
\chapterstyle{combined}
\tableofcontents*


% Starting the real text.
\cleardoublepage
\pagenumbering{arabic}
\setcounter{page}{1}


%\cftaddnumtitleline{toc}{chapter}{Preface}{}{}%
\cftaddnumtitleline{toc}{chapter}{Preface}{}{\arabic{page}}%
\chapter*{Preface}

Preface and acknowledgements if you like to have this.



\cleardoublepage
%\cftaddnumtitleline{toc}{chapter}{Preface}{}{}%
\cftaddnumtitleline{toc}{chapter}{List of Abbreviations}{}{\arabic{page}}%
\chapter*{List of Abbreviations}

....


%%%%%%%%%%%%%%%%%%%%%%%%%%%%%%%%%%%%%%%%%%%%%%%%%%%%%%%%%%%%%%%%%%%%%%%%%%%%%%%%
%%% INTRODUCTION
%%%%%%%%%%%%%%%%%%%%%%%%%%%%%%%%%%%%%%%%%%%%%%%%%%%%%%%%%%%%%%%%%%%%%%%%%%%%%%%%
\cleardoublepage

\chapter{Introduction}

%\section{Section}

%\subsection{Subsection}

\section{Terminology}

directly reversible = an injective function

%\part{}

\chapter{Implementing cryptographic algorithms in reversible languages}

In this chapter we show the implementation of several cryptographic 
algorithms and discuss the advantages reversibility brings.

\section{Implementations}

%\section{Advantages}

\section{Lack of state information leakage from data remanence}

A major advantage over non-reversible languages is the required 
clean-up of intermediate results. For an example, consider the 
\name{Simon} (NSA) algorithm. The algorithm takes a key and some 
plaintext as the input and is expected to produce ciphertext on the 
output. All other temporarily allocated data should be freed 
deallocated again. The Simon algorithm expands the key into a longer 
string, an \term{expanded key}, which is then used for the encryption 
itself. After the encryption is done, the intermediate expanded key is 
supposed to be wiped and deallocated. The issue with irreversible 
languages is that they cannot ensure the wiping of the expanded key. It 
is perfectly legal to only deallocate the memory, which can then be 
mapped by another process, making it readable. 

Implementing the algorithm in a reversible language leads to either 
wiping the expanded key, or leaking it by intention, \ie making it into 
a valid output. Unless the second option is favoured for some reason, 
we can uncall the key expansion, thus wiping the expanded key. 
Furthermore, in the extended version of Janus this becomes even more 
pronounced. Whatever is allocated as a local variable also has to be 
deallocated in the same scope. Therefore, by making the expanded key 
into a local variable, we are forced to wipe its state into a known 
value, presumably an array of zeroes.

We can observe this in the code. The expanded key is allocated at the 
beginning of the encryption and deallocated at the end. So far this is 
expected of any implementation, regardless of reversibility, thanks to 
common software engineering techniques like \term{RAII}. After other 
local variable declarations, the key expansion is performed, followed 
by the encryption itself. Just before deallocating local variables, we 
are forced to uncall the key expansion to get rid of the extra data. 
Reversibility forces us to provide the final value of all local 
variables, which, in the reverse execution, is also the initial value.

\section{Duality of encryption/decryption}

The second advantage is the fact decryption is the reverse of 
encryption. There is no reason to implement decryption functions, the 
encryption can be simply uncalled. The lack of decryption function has 
several implications:

\begin{itemize}

\item less time spent writing and debugging the code itself

% incorrectly different implementation of either encryption or decryption
\item less time spent debugging caused by errorneous implementation of 
either encryption or decryption, leading to one being not the reverse 
of the other

\item less code to maintain

\item smaller library size

\end{itemize}

In any software project, these are valuable features.

\section{Other side channels}

\todo{is this section useful?}

Unfortunately, due to the unavailability of real physical reversible 
circuits, we can only speculate about other side-channel 
vulnerabilities. One common side channel is the timing differences in 
execution that is input-dependent. One example are bit rotations. "Bit 
rotation by $x$ bits" can be implemented by either directly rotating by 
$x$ bits, or by $x$ rotations by 1 bit. This difference made RC5 
vulnerable to timing-based attack on some platforms\cite{TimingRC5}. 
Since bit rotations are directly reversible, there is no difference in 
comparison to irreversible implementations.

A directly irreversible operation is multiplication, assuming 
word-sized operations with modulo semantics. When implemented, it has 
to make use of the Bennett's trick\todo{I think this is right, but I'm tired.}. This implies that it will execute 
once to calculate the result, and once to wipe it. If it is 
timing-variable, the difference will only get more distinct in the 
execution time of the program.

Another side-channel vulnerability is variable power consumption. 
However, given the fact entropy stays the same throughout the 
computation, thus no energy is wasted, energy is used only by the 
supporting hardware. Ideally, this would imply no information can be 
gathered about the data within the system. Power consumption analysis 
is a viable attack against devices which the attacker has full physical 
control over, and it is an issue to be considered when designing, \eg, 
smart cards. Still, without actual physical implementation, this can 
only be a speculation. \todo{this does sound too good to be that simple 
and true}

\chapter{Practical programming experience in reversible languages}

\section{How to}
\todo{bad section name}

There are two general ways of implementing any algorithm in a 
reversible language:
\begin{enumerate}

\item Implementing it directly and manually reversing anything that 
produced intermediate data which we need to remove. This was usually 
the case with the actual encryption, as it tends to be straightforward 
in this case.

\item If any part of that algorithm is difficult to manually reverse, 
we can implement that part of the algorithm in its own procedure, while 
its output is stored in locally allocated variables by its caller. 
After the use of those values has passed, we uncall the algorithm, 
which resets the values in those variables, allowing for deallocating 
them. This is called the \term{Bennett's trick}.

\end{enumerate}

\section{Already implemented improvements}

\begin{itemize}

\item 

\end{itemize}

\section{Suggested further improvements}

\begin{itemize}

\item Constants that do not require explicit deallocation. Macros are fine.

\item Functions with return values. These can be implemented using a 
hidden local variable with initialization to the function call and 
deinitialization by function uncall.

\end{itemize}

\chapter{Conclusion}

Discussion of results.


%%%%%%%%%%%%%%%%%%%%%%%%%%%%%%%%%%%%%%%%%%%%%%%%%%%%%%%%%%%%%%%%%%%%%%%%%%%%%%%%
%%% BIBLIOGRAPHY
%%%%%%%%%%%%%%%%%%%%%%%%%%%%%%%%%%%%%%%%%%%%%%%%%%%%%%%%%%%%%%%%%%%%%%%%%%%%%%%%

\cleardoublepage
\renewcommand{\sc}[1]{\textsc{#1}}
\nocite{*}
\bibliographystyle{acm}
\bibliography{bibliography}


%%%%%%%%%%%%%%%%%%%%%%%%%%%%%%%%%%%%%%%%%%%%%%%%%%%%%%%%%%%%%%%%%%%%%%%%%%%%%%%%
%%% Appendix
%%%%%%%%%%%%%%%%%%%%%%%%%%%%%%%%%%%%%%%%%%%%%%%%%%%%%%%%%%%%%%%%%%%%%%%%%%%%%%%%
\cleardoublepage
\appendix

% Updates section names to remove . (e.g. a section is named A1 instead of A.1)
\renewcommand{\thesection}{\thechapter\arabic{section}}

\chapter{The extra}


% To make sure that the backside print of the dissertation is a white page
\newpage ~
\thispagestyle{empty}



\end{document}
